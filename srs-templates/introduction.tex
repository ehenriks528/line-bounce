\chapter{Introduction}
\section{Background }

The casual gaming industry is huge and is responsible for supplying
over 200 million people worldwide with entertainment.The purpose of
Line Bounce is to supply people with temporary entertainment in a
safe environment. Since Line Bounce is so closely related to Doodle
Jump, it might be beneficial to talk about the success of Doodle Jump
in order to express our hope for Line Bounce.

Doodle Jump has sold more than 3.5 million copies of its game for
users on cell phones, iPads, and other devices. This accomplishment
has made the developers of Doodle Jump the world\textquoteright{}s
most successful iPhone and iPod touch developers. It\textquoteright{}s
a simple but addicting game that has easily provided the developers
over a million dollars before taxes and excluding Apple\textquoteright{}s
30\% cut. With such success from a similar game, there is a hope that
Line Bounce will be similarly successful.


\section{Concept}

The concept of the game is to have an avatar jump as high as possible
using platforms that will be drawn by the player. There will be enemies
in the path of the avatar, so the player will need to draw angled
platforms to dodge the enemies. Also, the player can angle their avatar\textquoteright{}s
path to obtain power-ups and coins in the environment. The levels
will also have a limitless height and the user\textquoteright{}s avatar
can use the walls to bounce off of as well. In addition, the player
will be rewarded with coins after every 1000 feet they clear. The
environments can be changed from game to game to keep the player entertained. 


\section{Purpose}

The purpose of this document is to clearly outline the client requirements
and the system requirements for the game Line Bounce. This document
should cover all key concepts and and fundamental requirements. The
requirements should both satisfy the client\textquoteright{}s wishes
in the game and provide accuracy and functionality for the game.


\section{Intended Stakeholders }
\begin{itemize}
\item Team Members: To completely understand what will be needed for implementation
when developing Line Bounce. 
\item Project Client: To ensure the designed game meets the client\textquoteright{}s
requirements. 
\item Supervisor: To understand what the project entails and entrust the
group with implementation of said project. 
\item Maintainers: For those who will maintain the game after launched,
they may refer to this document to gain insight of the system and
an understanding of the project goals. 
\item Future Developers: For the future developers, they may look at this
document to gain insight into the original game design in order to
develop and enhance the game appropriately.
\end{itemize}

\section{Scope }
\begin{itemize}
\item All requirements for the game will be clearly explained and expressed
in this document. 
\item This document states all requirements, functional or non-functional. 
\item When requirements meet the standard of the client, the client will
sign the document as an agreement to the given requirements for the
game Line Bounce.\end{itemize}
